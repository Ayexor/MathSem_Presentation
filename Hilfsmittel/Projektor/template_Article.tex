\documentclass[]{article}
\usepackage{amsmath}

%opening
\author{Stfan Kull, Roy Seitz}
\title{Analytische Fortsetzung}


\begin{document}
	
\maketitle{}

\huge

\begin{tabular}{|l|l|}
	\hline
	$w^+=\begin{pmatrix}
	a_{11} & a_{12} \\ a_{21} & a_{22}
	\end{pmatrix}$ w&
	$\begin{matrix}
		w_1^{+}&=&a_{11}w_1&+&a_{12}w_2\\			w_2^{+}&=&a_{21}w_1&+&a_{22}w_2
	\end{matrix}$ \\ 
	
	\hline
	$w^+=\begin{pmatrix}
		\lambda_1 & 0 \\ 0 & \lambda_2
	\end{pmatrix}w $&
	
	$\begin{matrix}
		w_1^{+}&=&\lambda_1w_1&\\
		w_2^{+}&=&&&\lambda_2w_2
	\end{matrix}$ \\
	
	\hline	 
	$w^+=\begin{pmatrix}
	\lambda & 0 \\ 1 & \lambda
	\end{pmatrix}w$&
	
	$\begin{matrix}
		w_1^{+}&=&\lambda_1w_1&\\
		w_2^{+}&=&w_1&+&\lambda_2w_2
	\end{matrix}$\\
	\hline
\end{tabular}
\end{document}
